\chapter{Avancement de la construction système}

\section{Tests}
La partie logicielle comporte bien sûr des tests unitaires qui assurent la maintenabilité et la clareté du code.
Les tests fonctionnels peuvent être effectués à l'aide d'un simulateur qui remplace le robot physique.
On peut paramétrer ce simulateur afin qu'il induise du bruit dans le système et soit donc ainsi plus difficile à contrôler.
Ce simulateur est contrôlé par la vraie interface utilisateur du système.
Il suffit de changer le robot dans un fichier de configuration.
Ce fichier de configuration nous permet d'injecter différentes composantes selon le contexte d'exécution.

\section{Interface}
Afin d'obtenir une interface modulaire et la plus polyvalente possible nous avons opté pour une interface web.
Une composante de notre système s'occupe donc de servir les fichiers nécessaires à l'application web.
Celle-ci communique par HTTP REST ou par socket aux autres composantes du système.
La communication par socket est utilisée lorque nous avons besoin d'une boucle de rétroaction rapide,
comme dans le cas de l'information de position du robot.
La section représentant le monde de jeu de l'interface peut afficher beaucoup d'informations.
Afin de ne pas la surcharger, nous avons opté pour que chacun des affichages soit optionnel et puisse donc
être ajouté ou enlevé indépendamment, formant ainsi différentes couches.
Du point de vue des technologies, nous avons utilisé Bootstrap et AngularJS de manière à construire rapidement.
Pour dessiner les différents éléments de la carte, nous utilisons EaselJS qui nous permet d'utiliser dess fonctions avancées de dessin.
