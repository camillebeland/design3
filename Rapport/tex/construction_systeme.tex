\chapter{Avancement de la construction système}

\section{Système de vision}

Le framework Open CV est utilisé pour appliquer des filtres et des traitements permettant de détecter le contenu de la carte soit 
les îles, les trésors, la station de recharge et la position du robot. La création de fonctions utilitaires de type pipe and filter 
permet de tester plus facilement quels filtres sont efficaces et à quel moment, car l'image initiale n'est pas modifiée dans le 
traitement. Il est donc possible de voir l'effet de chaque filtre à tout moment. De plus, nous avons créé une configuration 
pour les paramètres qui permet de les modifier et de facilement détecter lesquels sont les plus efficaces. Éventuellement, des
tests automatisés seront réalisés avec un grand échantillon d'images afin d'ajouter de la robustesse à nos algorithmes de détection.
Une fois la position des objets détectée, elle est envoyée au robot qui crée la grille utilisée dans la recherche de chemin.


La caméra monde est utilisée au début pour détecter les objets de la carte et construire la grille pour la recherche de chemin. Elle envoie aussi à intervalle régulier la position du robot au robot. La caméra embarquée, quant à elle, est utilisée pour avoir des précisions sur le positionnement précis du robot près de la station de recharge, avant de ramasser le trésor et avant de déposer le trésor.