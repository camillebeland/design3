\section{Transmission et décodage du code manschester}

Afin de transmettre le code manschester entre la station de base et le robot, un module transmetteur et un récepteur de fréquence radio 433 MHz sont utilisée. Pour ce faire, un microcontroleur de type Arduino Mega se trouvant à la station de base lit d'abord le code manschester, sans le décoder, puis diffuse en continu le signal de données sur la fréquence utilisée avec un baudrate de 600. Ainsi, lorsque le robot doit décoder le signal, il ne fait qu'écouter sur cette fréquence, avec un baudrate de 600, puis lire les 32 premier bits recus.
\paragraph{}
Le décodage se fait ensuite purement de manière logiciel par le microcontroleur du robot. Puisque le signal d'horloge n'est pas tranmit, il faut trouver le point de départ en identifiant la séquence de 18 bits fixes de départ (le signal de données manschester contient deux fois plus de bits que le code original). Une fois cette séquence de départ identifiée, il suffit de décoder les 14 bits restants. Pour ce faire, les bits sont simplement regroupés en sous-séquences de deux bits. Une sous-séquence 0-1 signifie un '1' tandis qu'une sous-séquence 1-0 signifie un '0'. Tout autre sous-séquence est interprétée comme une erreur et on recommence l'acquisition du code manschester. Suite à ce décodage, une séquence de 7 bits est obtenue, correspondant au signal ASCII désiré.