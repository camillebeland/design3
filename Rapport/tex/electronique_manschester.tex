\section{Transmission et décodage du code manschester}

Afin de transmettre le code manschester entre la station de base et le robot, un module transmetteur et un récepteur de fréquence radio 433 MHz sont utilisée. Pour ce faire, un microcontroleur de type Arduino Mega se trouvant à la station de base lit d'abord le code manschester en se synchronisant par interuption sur le signal de l'horloge, sans le décoder. Par la suite, un port TX est utilisé afin de diffuser en continue le code avec un baudrate de 600 dans le transmetteur RF. Sur le robot, le récepteur RF est brancher sur un port RX avec le même baudrate que le transmetteur. Il suffit alors de lire le contenu du buffer série 4 fois (4x8 bits) afin d'obtenir les 32 bits du code manchester qui est décodé par la suite sur le Arduino. Le Arduino décode continuellement le contenu du buffer série.
\paragraph{}
Puisque le signal d'horloge n'est pas transmit, il faut trouver le point de départ en identifiant la séquence de 18 bits fixes de départ (le signal de données manschester contient deux fois plus de bits que le code original). Une fois cette séquence de départ identifiée, il suffit de décoder les 14 bits restants. Pour ce faire, les bits sont simplement regroupés en sous-séquences de deux bits. Une sous-séquence 0-1 signifie un '1' tandis qu'une sous-séquence 1-0 signifie un 0. Tout autre sous-séquence est interprétée comme une erreur et on recommence l'acquisition du code manschester. Suite à ce décodage, une séquence de 7 bits est obtenue, correspondant au signal ASCII désiré. Si le contenu ne correspond pas à un code manchester (même nombre de 1 que de 0, 8 bits à "1" de suite suivi d'un bit à "0"), le dernier code manchester valide est gardé en mémoire jusqu'à ce qu'un autre code valide le remplace. L'ordinateur embarqué peut en tout temps demandé le dernier code manchester décodé valide.
